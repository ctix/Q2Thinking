\documentclass{article}
\usepackage{amsmath}
\usepackage{amsfonts}
\usepackage{amssymb}
\usepackage{graphicx}
\usepackage[utf8]{inputenc}
\usepackage{geometry}
\geometry{a4paper, margin=1in}

\title{Model Compression Techniques: Distillation and Pruning for Efficient Deep Learning}
\author{AI Assistant (Jules)}
\date{\today}

\begin{document}
\maketitle
\begin{abstract}
This document outlines the principles of model distillation and pruning as techniques for reducing the size and computational cost of large neural network models, enabling their deployment in resource-constrained environments. It also touches upon typical lightweight Transformer architectures and the mathematical foundations of the associated loss functions. This work is in the context of creating a smaller 'student' model from a larger 'teacher' model, specifically for tasks related to electronic technology design.
\end{abstract}

\section{Introduction}
Large-scale neural networks, while achieving state-of-the-art performance in various domains, often come with prohibitive computational demands and memory footprints. Model compression techniques aim to alleviate these issues. This document focuses on two such popular methods: knowledge distillation and network pruning.

\section{Knowledge Distillation}
Knowledge distillation is a model compression technique where a smaller 'student' model is trained to mimic the behavior of a larger, pre-trained 'teacher' model. The core idea is to transfer the 'knowledge' learned by the teacher model to the student model.

\subsection{Teacher-Student Paradigm}
The teacher model is typically a high-capacity model that has been trained on a large dataset. The student model has a significantly smaller architecture (fewer parameters, layers, etc.). The goal is for the student to learn not just the ground truth labels but also the nuanced output distribution of the teacher.

\subsection{Knowledge Transfer Mechanisms}
\subsubsection{Soft Labels (Logits Distillation)}
Instead of using hard labels (one-hot encoded ground truth) for training the student, distillation often uses the softened outputs (logits before softmax, or probabilities after softmax with temperature scaling) from the teacher model as targets. The teacher's logits provide richer information about inter-class similarities.
The distillation loss, often using Kullback-Leibler (KL) divergence, measures the difference between the teacher's and student's output distributions.
The probability $p_i$ for a class $i$ is calculated using a softmax function with temperature $T$:
$$ p_i = \frac{\exp(z_i / T)}{\sum_j \exp(z_j / T)} $$
where $z_i$ are the logits. A higher temperature $T > 1$ softens the probability distribution, allowing the student to learn from smaller logit values.

\subsubsection{Feature Distillation}
Knowledge can also be transferred by encouraging the student model's intermediate layer representations (features) to be similar to those of the teacher model. This can help the student learn the teacher's internal reasoning process.

\section{Network Pruning}
Network pruning aims to reduce model size by removing redundant or less important parameters, neurons, or even entire structural components from a trained network.

\subsection{Types of Pruning}
\subsubsection{Unstructured Pruning}
Individual weights are removed based on some criterion (e.g., magnitude). This can lead to sparse weight matrices that require specialized hardware or libraries for efficient inference.
\paragraph{Magnitude Pruning:} Weights with absolute values below a certain threshold are set to zero. L1 regularization during training is a common way to encourage weights to become small, effectively pruning them. The L1 penalty added to the loss function is:
$$ L_{L1} = \lambda \sum_i |\theta_i| $$
where $\theta_i$ are the model parameters and $\lambda$ is the regularization strength.

\subsubsection{Structured Pruning}
Entire neurons, channels, or layers are removed. This results in a smaller, dense model that can be run efficiently on standard hardware.

\subsection{Layer Freezing}
A specific form of structured pruning or model adaptation where entire layers, particularly those deemed less relevant to the target task, are frozen (their weights are not updated during fine-tuning or distillation). This is common when adapting a large pre-trained model to a specific downstream task, preserving general knowledge in lower layers while specializing upper layers.

\section{Lightweight Transformer Architecture for Student Model}
For tasks like natural language understanding and generation, the student model is often a smaller version of the Transformer architecture.
\begin{itemize}
    \item \textbf{Reduced Layers:} Fewer encoder and/or decoder layers compared to the teacher.
    \item \textbf{Smaller Hidden Dimensions:} Reduced size for embedding layers, feed-forward network hidden layers, and attention heads.
    \item \textbf{Fewer Attention Heads:} A smaller number of attention heads in the multi-head attention mechanism.
    \item \textbf{Shared Parameters:} Techniques like Albert (A Lite BERT) share parameters across layers to reduce redundancy.
    \item \textbf{Alternative Architectures:} Models like DistilBERT, TinyBERT, or MobileBERT are designed specifically for efficiency. Gemma models also offer smaller variants (e.g., 2B parameters).
\end{itemize}
The goal is to maintain the core self-attention and feed-forward mechanisms of the Transformer while significantly reducing the parameter count to the desired target (e.g., ~1.5 billion parameters).

\section{Mathematical Principles of Loss Functions}
\subsection{Cross-Entropy Loss (Task Loss)}
For classification or generation tasks, the student model is often trained on the actual task labels. The cross-entropy loss for a single data point is:
$$ L_{CE} = - \sum_c y_c \log(p_c) $$
where $y_c$ is the true probability (1 for the correct class, 0 otherwise for hard labels) and $p_c$ is the student model's predicted probability for class $c$. For language modeling, this is calculated over the vocabulary at each token position.

\subsection{KL Divergence Loss (Distillation Loss)}
This measures the difference between the probability distribution of the teacher ($P_T$) and the student ($P_S$), typically after applying temperature scaling to their logits.
$$ L_{KD} = D_{KL}(P_T || P_S) = \sum_i P_T(i) \log\left(\frac{P_T(i)}{P_S(i)}\right) $$
When using soft targets from the teacher (probabilities $q_i$) and student predictions ($p_i$), both derived with temperature $T$, the loss is often scaled by $T^2$:
$$ L_{Distill} = (T^2) \times \sum_i q_i \log\left(\frac{q_i}{p_i}\right) $$

\subsection{Combined Loss}
The total loss function during distillation is typically a weighted sum of the task loss and the distillation loss:
$$ L_{Total} = \alpha \cdot L_{Distill} + (1 - \alpha) \cdot L_{CE} $$
where $\alpha$ is a hyperparameter balancing the two terms.

\subsection{L1 Regularization Penalty (Pruning)}
As mentioned earlier, the L1 penalty added to the total loss to encourage sparsity is:
$$ L_{L1} = \lambda \sum_i |\theta_i| $$
So the final training loss becomes:
$$ L_{Final} = L_{Total} + L_{L1} $$

\section{Conclusion}
Model distillation and pruning are powerful techniques for creating efficient models that retain significant capabilities of larger counterparts. By carefully selecting the student architecture, distillation objectives, pruning strategies, and loss functions, it's possible to achieve substantial compression while preserving performance on targeted tasks, as demonstrated by the goal of creating a specialized 1.5B parameter model for electronic technology design.

\end{document}
