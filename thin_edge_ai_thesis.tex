\documentclass{article}
% \usepackage[UTF8]{ctex} % Use ctex for Chinese characters - replaced by CJKutf8
\usepackage{CJKutf8}
\usepackage[backend=biber, style=numeric]{biblatex}
\addbibresource{references.bib}
\title{Thin-Edge AI:开启智能设备新纪元}
\author{Your Name} % Replace with actual author if known, otherwise leave as placeholder
\date{\today}

\begin{document}
\begin{CJK*}{UTF8}{gkai} % Start CJK environment with AR PL UKai
\maketitle

\begin{abstract}
Thin-Edge AI(新锐概念)作为一种创新的轻量级 AI 部署模式,正逐渐崭露头角。这一概念的核心在于利用现有的消费电子设备,如智能手机,作为边缘节点,而非传统的专用边缘服务器。这种模式不仅降低了部署成本,还提高了系统的灵活性和可扩展性,使得 AI 技术能够更广泛地应用于各种场景。
\end{abstract}

\section{引言}
在当今快速发展的科技时代,Thin-Edge AI(新锐概念)作为一种创新的轻量级 AI 部署模式,正逐渐崭露头角。这一概念的核心在于利用现有的消费电子设备,如智能手机,作为边缘节点,而非传统的专用边缘服务器。这种模式不仅降低了部署成本,还提高了系统的灵活性和可扩展性,使得 AI 技术能够更广泛地应用于各种场景。

\section{Thin-Edge AI 技术框架}
Thin-Edge AI 的技术框架可以概括为三个关键环节:Sense(感知)、Inference(推理)和 Augment(增强)。首先,Sense 阶段通过各种传感器(如扫地机的传感器、玩具麦克风等)采集数据,这些数据是 AI 系统运行的基础。接着,在 Inference 阶段,手机等消费电子设备运行轻量级的 AI 模型(如 7B 或 10B 模型),处理核心逻辑,实现快速响应和本地决策。最后,在 Augment 阶段,云端处理复杂的案例,并根据反馈更新模型,确保系统的持续优化和改进。

\section{数据印证}
这种模式的数据印证点在电子宠物场景中表现得尤为明显。根据联发科白皮书的数据~\cite{mediatek_whitepaper},本地模型能够处理 92% 的常规交互,而仅有 17% 的查询需要云端协同。这一数据充分证明了 Thin-Edge AI 在处理日常任务时的高效性和自给自足能力,同时也展示了云端在处理复杂任务时的重要补充作用。

\section{Thin-Edge AI 的创新之处}
Thin-Edge AI 的创新之处在于将智能手机定位为“AI 能力中转站”,这一角色不仅充分利用了智能手机强大的计算能力和广泛的普及度,还为 AI 技术的广泛应用提供了新的思路。从开源大模型到专用小模型的转化路径,使得 AI 技术能够更好地适应不同的应用场景,提高了模型的实用性和效率。

\section{技术趋势符合性}
这种模式符合当前的“On-Device AI”(设备端 AI)和“Hybrid Cloud”(混合云)的技术趋势。设备端 AI 强调在本地设备上进行数据处理和决策,减少对云端的依赖,提高响应速度和数据隐私保护。混合云则结合了本地计算和云端计算的优势,实现了资源的灵活调配和高效利用。

\section{结论}
综上所述,Thin-Edge AI 不仅是一种创新的 AI 部署模式,更是一种符合未来发展趋势的技术理念。它通过将智能手机等消费电子设备作为边缘节点,实现了高效的数据处理和智能决策,为各种应用场景提供了强大的技术支持。随着技术的不断进步和应用场景的不断拓展,Thin-Edge AI 必将在未来的智能设备领域发挥越来越重要的作用。

\CJKfamily{gkai} % Ensure CJK font is active for bibliography
\printbibliography
\end{CJK*} % End CJK environment
\end{document}
